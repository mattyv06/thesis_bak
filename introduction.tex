\chapter{Introduction}\label{CH:introduction}
Welcome to the Montana State University electronic Thesis/Dissertation (ETD) \LaTeX{} template.  In this chapter various sections, subsections, and subsubsections are created and filled with random text).  In Ch.~\ref{CH:theory} methods to write equations and how to include figures and tables are explored. Conclusions are drawn in Ch.~\ref{conclusion}.


\section{Section}\label{Sect:test}
\lipsum[1] % Random text

\subsection{Subsection}\label{Sect:testsub}
\lipsum[2] % Random text

\subsubsection{Subsubsection}\label{Sect:testsubsub}
\lipsum[3] % Random text

\longsubsection{Subsection With a Very Very Very Very}{Very Very Very Very Very Very Long Title}\label{Sect:longsub}
For long subsection titles use the command \verb|\longsubsection{#1}{#2}|, where \#1 is the first line of the long title, and \#2 is the second line of the long title. You can also pass an optional argument to this command that puts a shorter title in the table of contents as shown by the subsection below.

\longsubsection[Another Subsection With a Very Long Title]{Another Subsection With a Very Very Very}{Very Very Very Very Very Very Long Title}\label{Sect:longsub2}
The are \textbf{not} similar commands for sections and subsubsections as these are not specified in the MSU style guide.  

\chapter{Introduction}

\newcount\loopcount
\loopcount 5
\loop
  \section{Section}
  \lipsum[1] % Random text

  \subsection{Subsection}
  \lipsum[2] % Random text

  \subsubsection{Subsubsection}
  \lipsum[3] % Random text

\advance\loopcount-1
\ifnum\loopcount>0
\repeat 